\documentclass[a4paper]{article}

\usepackage[portuguese]{babel}
\usepackage{comment}
\usepackage[T1]{fontenc}
\usepackage[utf8]{inputenc}
\usepackage{hyperref}
\usepackage{graphicx}
\usepackage{float}
\usepackage{multirow}
\usepackage[hypcap]{caption} % makes \ref point to top of figures and tables
\usepackage{amsmath}
\usepackage[usenames,dvipsnames,svgnames,table]{xcolor}
\usepackage{rotating}

\begin{document}

\begin{titlepage}

	\begin{center}

		\includegraphics[width=6cm]{./title}\\[3cm]

		\textsc{\LARGE Segurança Informática em Redes e Sistemas}\\[1.5cm]

		\textsc{\Large Projecto}\\[1.5cm]


		{ \huge \bfseries Zero-Day Vulnerability \\[2.5cm] }


		\noindent
		\begin{center} \large
			Gonçalo Ribeiro, 73294\\[5mm]

			António Bacelar de Sousa, 73425\\[5mm]

			Rafael Gonçalves, 73786\\[2.5cm]

		\end{center}

		\begin{minipage}{0.4\textwidth}
			\begin{flushleft} \Large
				Prof. Ricardo Chaves
			\end{flushleft}
		\end{minipage}
		\begin{minipage}{0.4\textwidth}
			\begin{flushright} \Large
				Prof. Miguel Pardal
			\end{flushright}
		\end{minipage}

		\vfill

		{\large \today}

	\end{center}

\end{titlepage}

\tableofcontents
\pagebreak

\section{Motivação}

\pagebreak

\section{Objectivos}
\label{sec:objectivos}

\begin{itemize}
	\item Básico: especificar uma vulnerabilidade no Haihaisoft Universal Player
	\item Intermédio: realização de um ataque ao software por intermédio da vulnerabilidade especificada
	\item Avançado: investigação e aplicação de métodos que permitam resolver total ou parcialmente a vulnerabilidade
\end{itemize}

\subsection{Básico}



\subsection{Intermédio}

\subsection{Avançado}

\pagebreak

\section{Plano de Trabalho}

\subsection{Semana 1}
Prazo: 07/11/14
\subsubsection{Disponibilidade}
Reduzida devido a avaliações e entregas de projectos de outras UCs.

\subsubsection{Tarefas}
\begin{itemize}
\item escolher um software/website com uma vulnerabilidade activa
\item definir os parâmetros iniciais do projecto
\end{itemize}

\subsubsection{Trabalho Realizado}
Nesta semana procurámos listas de software/websites com vulnerabilidades activas e escolhemos o Haihaisoft Universal Player como objecto de estudo deste projecto. Este software é desenvolvido para Windows e é software aberto.

Foram também delineados os objectivos deste projecto (Secção~\ref{sec:objectivos}).

\subsection{Semana 2}
Prazo: 14/11/14
\subsubsection{Disponibilidade}
Reduzida devido a avaliações e entregas de projectos de outras UCs.

\subsubsection{Tarefas}
\begin{itemize}
\item preparar uma máquina virtual que possa ser usada ao longo do projecto
\item testar a \textit{proof of concept} (POC) da vulnerabilidade zero-day
\end{itemize}

\subsubsection{Trabalho Realizado}
Tendo-se verificado que o Haihaisoft Universal Player funciona exclusivamente em Windows foi esse o sistema operativo escolhido para instalar na máquina virtual. Como o Windows é software pago teve-se em conta esse factor de forma a fazer uma instalação que não tivesse problemas com licenças. Neste sentido e visto que o Windows 10 Technical Preview foi lançado recentemente foi esta a versão do Windows que se escolheu instalar na máquina virtual. De seguida instalou-se o Haihaisoft Universal Player e verificou-se o seu correcto funcionamento.

Uma das coisas que nos impressionou foi o facto de que o Haihaisoft Universal Player precisa de privilégios de administrador para ser executado. Como tal, e visto que este software é vulnerável a buffer overflows, o código que conseguirmos injectar vai correr com privilégios de administrador deixando a máquina largamente à nossa mercê.

Por fim testámos a POC encontrada na Exploit Database\footnote{http://www.exploit-db.com/exploits/32514/}. Esta POC consiste num pequeno script escrito em Python. Como tal instalámos Python na máquina virtual. Tentámos correr o script mas o Python revelava alguns erros. Após correcção desses erros conseguimos gerar 3 ficheiros de exploit. Experimentámos então abrir esses ficheiros no Haihaisoft. Os resultados dos 3 ficheiros foram semelhantes. Com um deles o programa fechou-se mal era aberto. Com os outros dois ficheiros o programa deixava de responder (aparecia o menu do Windows a dar essa informação) e passado algum tempo o programa era fechado. O shellcode introduzido é apenas uma instrução, que ainda não tivemos oportunidade de verificar o que faz. No entanto verificámos que claramente algo de errado acontece com o programa ao abrir os ficheiros de exploit.

\subsection{Semana 3}
Prazo: 21/11/14
\subsubsection{Disponibilidade}
Reduzida devido ao teste de SIRS.

\subsubsection{Tarefas}

\subsection{Semana 4}
Prazo: 28/11/14
\subsubsection{Disponibilidade}
\subsubsection{Tarefas}

\subsection{Semana 5}
Prazo: 05/12/14
\subsubsection{Disponibilidade}
\subsubsection{Tarefas}

\pagebreak

\section{Resultados}

\subsection{Esperados}
\subsection{Obtidos}

\pagebreak

\section{Referências}

\end{document}