\documentclass[11pt,a4paper]{article}

\usepackage[portuguese]{babel}
\usepackage[T1]{fontenc}
\usepackage[utf8]{inputenc}
\usepackage{hyperref}
\usepackage{graphicx}
\usepackage{float}
\usepackage[hypcap]{caption} % makes \ref point to top of figures and tables
\usepackage{amsmath}
\usepackage[nottoc,numbib]{tocbibind}	% adds Biliography to index
\usepackage{listings}
\usepackage[margin=2.3cm]{geometry}

\begin{document}

\pagenumbering{gobble}	% disable page numbering
\begin{titlepage}

	\begin{center}

		\includegraphics[width=6cm]{./title}\\[3cm]

		\textsc{\LARGE Segurança Informática em Redes e Sistemas}\\[1.5cm]

		\textsc{\Large Projecto}\\[1.5cm]


		{ \huge \bfseries Zero-Day Vulnerability \\[2.5cm] }


		\noindent
		\begin{center} \large
			Gonçalo Ribeiro, 73294\\[5mm]

			António Bacelar de Sousa, 73425\\[5mm]

			Rafael Gonçalves, 73786\\[2.5cm]

		\end{center}

		\begin{minipage}{0.4\textwidth}
			\begin{flushleft} \Large
				Prof. Ricardo Chaves
			\end{flushleft}
		\end{minipage}
		\begin{minipage}{0.4\textwidth}
			\begin{flushright} \Large
				Prof. Miguel Pardal
			\end{flushright}
		\end{minipage}

		\vfill

		{\large \today}

	\end{center}

\end{titlepage}

\tableofcontents
\pagebreak

\pagenumbering{arabic}
\section{Motivação}
Este trabalho tem como motivação fazer um estudo aprofundado de uma \textit{zero-day vulnerability}, e, desta forma, perceber como funcionam este tipo de ataques e como são tratadas as vulnerabilidades a eles associadas. Para além disso, como motivação também surge o facto de esta ser uma oportunidade para aplicar os conhecimentos adquiridos este semestre na disciplina de Segurança Informática em Redes e Sistemas.

\section{Objectivos}
\label{sec:objectivos}

\subsection{Antes de 25 de Novembro}

\subsection*{Básico}
Especificar uma vulnerabilidade no Haihaisoft Universal Player.
\subsection*{Intermédio}
Realização de um ataque ao software por intermédio da vulnerabilidade especificada.
\subsection*{Avançado}
Investigação e aplicação de métodos que permitam resolver total ou parcialmente a vulnerabilidade do Haihaisoft UP.

\subsection{Após 25 de Novembro}
Após uma análise cuidada e tentativas de ataque ao Haihaisoft UP chegou-se à conclusão de que era inviável fazer uma \textit{exploit} que fosse consistente entre versões do Windows. Vimo-nos portanto obrigados a escolher outro software com uma vulnerabilidade \textit{zero-day}. De forma a aproveitar o conhecimento adquirido escolheu-se um \textit{software} também vulnerável a \textit{buffer overflows}: o i.FTP. Seguem-se os novos objectivos.

\subsection*{Básico}
Especificar uma vulnerabilidade no \textit{software} i.FTP.
\subsection*{Intermédio}
Realização de um ou vários ataques ao \textit{software} tirando proveito da vulnerabilidade especificada.
\subsection*{Avançado}
Tentar eliminar a vulnerabilidade estudada.

%\section{Plano de Trabalho}
\section{Trabalho Realizado}
\subsection{Semana 1}
%Prazo: 07/11/14
%\subsubsection{Disponibilidade}
%Reduzida devido a avaliações e entregas de projectos de outras UCs.

%\subsubsection{Tarefas}
%\begin{itemize}
%\item escolher um software/website com uma vulnerabilidade activa
%\item definir os parâmetros iniciais do projecto
%\end{itemize}

%\subsubsection{Trabalho Realizado}
Nesta semana procuraram-se listas de software/websites com vulnerabilidades activas e foi escolhido o Haihaisoft Universal Player\footnote{\url{http://www.haihaisoft.com/hup.aspx}}, um leitor universal de ficheiros multimédia, como objecto de estudo deste projecto. Este \textit{software} é desenvolvido para Windows e é \textit{software} aberto.

Foram também delineados os objectivos deste projecto (Secção~\ref{sec:objectivos}).

\subsection{Semana 2}
%Prazo: 14/11/14
%\subsubsection{Disponibilidade}
%Reduzida devido a avaliações e entregas de projectos de outras UCs.

%\subsubsection{Tarefas}
%\begin{itemize}
%\item preparar uma máquina virtual que possa ser usada ao longo do projecto;
%\item testar a \textit{proof of concept} (POC) da vulnerabilidade \textit{zero-day}.
%\end{itemize}

%\subsubsection{Trabalho Realizado}
Tendo-se verificado que o Haihaisoft Universal Player funciona exclusivamente em Windows, foi esse o sistema operativo escolhido para instalar na máquina virtual. Como o Windows é software pago teve-se em conta esse factor de forma a fazer uma instalação que não tivesse problemas com licenças. Neste sentido, e visto que o Windows 10 Technical Preview foi lançado recentemente, foi esta a versão do Windows que se escolheu instalar na máquina virtual. De seguida instalou-se o Haihaisoft UMP e verificou-se o seu correcto funcionamento.

Uma das coisas impressionantes foi o facto de o Haihaisoft UP precisar de privilégios de administrador para ser executado. Como tal, e visto que este software é vulnerável a buffer overflows, o código que se conseguir injectar vai correr com privilégios de administrador deixando a máquina largamente à mercê de potenciais atacantes.

Por fim foi testada a POC encontrada na Exploit Database\footnote{\url{http://www.exploit-db.com/exploits/32514/}}. Esta POC consiste num pequeno \textit{script} escrito em Python. Como tal, instalou-se Python na máquina virtual. Tentámos correr o \textit{script}, mas o código Python revelava alguns erros. Após correcção desses erros conseguiu-se gerar 3 ficheiros de \textit{exploit}. Experimentou-se então abrir esses ficheiros no Haihaisoft UP. Os resultados dos 3 ficheiros foram semelhantes: com um deles o programa fechou-se mal era aberto; com os outros dois ficheiros o programa deixava de responder (aparecia o menu do Windows a dar essa informação), e passado algum tempo o programa era fechado. O \textit{shellcode} introduzido é um conjunto de instruções cuja funcionalidade se desconhece. No entanto foi verificado que claramente algo de errado acontece com o programa ao abrir os ficheiros de \textit{exploit}.

\subsection{Semana 3}
%Prazo: 21/11/14
%\subsubsection{Disponibilidade}
%Muito reduzida devido ao teste de SIRS.

%\subsubsection{Tarefas}
%\begin{itemize}
%	\item Identificar o objectivo do código Assembly injectado através de \textit{buffer overflow}.
%\end{itemize}

%\subsubsection{Trabalho Realizado}
%Examinou-se o código fonte de maneira a descobrir o endereço de retorno, de maneira a alterar o código de exploit para correr código feito pelo grupo

Nesta semana foi analisada a POC. São escritos bytes com 4 valores diferentes para o ficheiro de \textit{exploit}. Chegou-se à conclusão de que os valores dos bytes se referem aos caracteres ASCII A, B e C e que o outro valor é usado como \textit{padding}.

\subsection{Semana 4}
%Prazo: 28/11/14
%\subsubsection{Disponibilidade}
%Maior disponibilidade.
%\subsubsection{Tarefas}
%\begin{itemize}
%	\item Desenvolvimento do código que se serve da vulnerabilidade encontrada para conseguir acesso com privilégios elevados.
%\end{itemize}

%\subsubsection{Trabalho Realizado}
A primeira conclusão a que se chegou foi que não é possível fazer um \textit{stack based overflow}. O Windows usa SEH (\textit{structured exception handlers}) o que faz com que \textit{stack based overflows} não sejam possíveis, ou seja, fazer um overwrite directo do EIP (instruction pointer) e fazer um RET.

Outra conclusão a que se chegou foi que o \textit{buffer} cuja capacidade é excedida não é directamente o \textit{buffer} que recebe a URL, mas sim um \textit{buffer} que contém a URL transformada, após transformação/eliminação de caracteres que não são permitidos em URLs. Isto significa que os valores de bytes que são possíveis escrever ficam muito mais limitados. Mais concretamente de 256 valores de bytes passa-se a poder usar apenas cerca de 90 valores. Por outro lado não é possível usar o valor \texttt{0x00} porque é o terminador de \textit{strings}, e portanto ao usar o programa este pára de copiar bytes assim que encontra um byte com esse valor (parando o overflow antes do que se quer).

O Haihaisoft UP não foi compilado com protecção da SEH. Ou seja em teoria é possível fazer \textit{overflow} das estruturas SEH e enganar o programa de forma a que escreva um novo valor no EIP, valor esse escolhido pelo grupo. Para fazer isto tem que se conseguir que o programa execute uma sequência de instruções POP POP RET, que tem que estar em código não compilado com protecção de SEH. Neste caso, o próprio executável não tem esta protecção pelo que esta sequência de instruções podia ser uma do próprio programa.

Os programas em Windows têm a memória dividida em 3 componentes. Entre estas, a componente de código é aquela que tem as instruções do programa. Ou seja, é lá que se tem que encontrar uma sequência POP POP RET. No entanto, verificou-se que os endereços de memória do segmento de código começam todos pelo byte \texttt{0x00}. Isto significa que para enganar o programa de forma a que este salte para a dita sequência de código, teria que se fazer um overwrite da SEH handler para um endereço cujo primeiro byte é \texttt{0x00}. Ora isso não é possível porque não é possível escrever esse valor de byte. Portanto, não se consegue que seja executada a sequência de instruções necessária para que se conseguísse alterar o EIP para uma zona de memória controlada pelo grupo de forma a executar uma \textit{exploit}.

Decidiu-se então alterar o alvo de estudo do nosso projecto do Haihaisoft UMP para outro \textit{zero-day}.


\subsubsection{Novo Paradigma do Projecto -- i.FTP 2.20}
Consultou-se uma vez mais a Exploit Database\footnote{\url{http://www.exploit-db.com}} de forma a encontrar um novo \textit{zero-day}. O candidato escolhido foi o i.FTP\footnote{\url{http://www.memecode.com/iftp.php}}. Trata-se de um cliente de FTP que apresenta vulnerabilidades de \textit{buffer overflow} (foi decidido manter o tipo de vulnerabilidade a ser explorada, de maneira a não desperdiçar o conhecimento adquirido).

O referido site oferece uma POC\footnote{\url{http://www.exploit-db.com/exploits/35177/}} para esta vulnerabilidade. Após análise dessa POC conclui-se que a vulnerabilidade resulta do carregamento de um ficheiro de configurações do programa -- Schedule.xml\footnote{localizado em C:\textbackslash Users\textbackslash someuser\textbackslash AppData\textbackslash Local\textbackslash VirtualStore\textbackslash Program Files (x86)\\ \textbackslash Memecode\textbackslash i.Ftp}. Existe um campo nesse ficheiro que se refere ao tempo para o qual se quer agendar uma transferência. Esse campo é susceptível a \textit{overflows}.

Foi feita uma análise da vulnerabilidade e conseguimos executar uma \textit{exploit} (abrir a calculadora do Windows). Esta \textit{exploit} não é muito interessante, pelo que começou também o desenvolvimento de uma \textit{exploit} que o fosse.

Ao correr o programa a calculadora é imediatamente aberta e o programa termina com a janela do Windows que diz que o programa não responde. Idealmente, o programa não terminaria inesperadamente, de maneira a preservar a ilusão de normalidade. A calculadora não é afectada pela terminação do i.FTP, ou seja a \textit{exploit} continua a correr mesmo com o fecho do i.FTP.


\subsection{Semana 5}
%Prazo: 05/12/14
%\subsubsection{Disponibilidade}
%Média.
%\subsubsection{Tarefas}
%\begin{itemize}
%	\item Elaborar um ataque mais interessante;
%	\item Evitar que o programa deixe de responder;
%	\item Tentar eliminar a vulnerabilidade;
%	\item Conclusão do presente documento.
%\end{itemize}

Decidiu-se aplicar uma \textit{exploit} que permitisse ter acesso remoto à máquina atacada. Para isso gerou-se um novo \textit{shellcode} com o seguinte comando:

\begin{lstlisting}
   ruby msfpayload windows/shell_bind_tcp EXITFUNC=seh
     LPORT=1234 R | ruby msfencode -a x86
     -b `\x00\x0a\x0d\x26' -t c
\end{lstlisting}

%	\texttt{ruby msfpayload windows/shell\_bind\_tcp EXITFUNC=seh \\ LPORT=1234 R | ruby msfencode -a x86 -b `\textbackslash x00\textbackslash x0a\textbackslash x0d\textbackslash x26' -t c}

que abre um porto TCP para a shell da máquina atacada no porto 1234. O atacante tem apenas que se ligar a esse porto para ter acesso à máquina:

\begin{lstlisting}
   nc a.b.c.d 1234
\end{lstlisting}

O resultado disto pode ser visto na Figura~\ref{fig:remote_shell}. Correndo o comando \texttt{whoami} é possível verificar que a \textit{shell} está a ser corrida como o user que lançou o i.FTP.

\begin{figure}[h]
        \centering
	\includegraphics[width=0.7\linewidth]{remote_shell}
	\caption{Acesso remoto à máquina atacada}
	\label{fig:remote_shell}
\end{figure}


\section{Resultados}



\subsection{Esperados}

Originalmente, os resultados esperados correspondiam à identificação, \textit{exploit} e correcção de uma vulnerabilidade no Haihaisoft UP. Uma vez determinado que estes resultados eram difíceis de conseguir, foi alterado o paradigma do projecto.

Neste novo âmbito, os resultados esperados passaram a ser outros.

Quanto à identificação da vulnerabilidade, era esperado conseguir os endereços relevantes na stack de maneira a desenhar uma estrutura para a \textit{exploit}.

Subsequentemente, após esta identificação, estava prevista a escrita da \textit{exploit}, de maneira a conseguir executar código injectado através da manipulação do endereço de retorno.

Finalmente, através de um varrimento do código-fonte da aplicação, seria aplicada uma correcção no código relativo ao \textit{buffer} alvo de ataque.

\subsection{Obtidos}

\subsubsection{Haihaisoft UP}

Uma vez analisada a POC disponível, foi determinado que era possível terminar o programa fora dos seus parâmetros de execução normais, fazendo \textit{overflow} de um \textit{buffer}. O \textit{buffer} em questão corresponde ao URL de um elemento de uma \textit{playlist}.

Rapidamente se concluiu que o texto era extraído com sucesso do ficheiro, e que só após uma transformação é que se dava o \textit{overflow}. Esta transformação corresponde à restrição dos caracteres aos que fazem parte do conjunto permitido em URL.

Esta restrição dos caracteres impossibilita uma escolha directa dos endereços para efectuar o salto.

Adicionalmente, não foi possível fazer um \textit{overwrite} directo do EIP (\textit{Extended Instruction Pointer}), sendo necessária a substituição de um elemento da lista de processamento de excepções, e consecutivo accionamento, de forma a modificar indirectamente o EIP.

A impossibilidade de inserir um endereço definido acabou por levar ao abandono deste software em específico, tendo sido seleccionada uma alternativa.

\subsubsection{i.FTP}

Ao analisar a POC, a determinação de endereços de salto e outros valores significativos mostrou-se impossível, pelo que se recorreu a ferramentas como o Immunity Debugger e o Metasploit. A determinação de endereços sem recurso a ferramentas auxiliares é uma técnica morosa e deprecada.

Uma vez feita esta análise (ver Anexo~\ref{sec:seh}), viu-se uma vez mais que a modificação directa do EIP é impossível, e é necessário a alteração das rotinas de processamento de excepções.

Acedendo à base de dados do Metasploit, construímos uma série de \textit{exploits} para o i.FTP com variadas funções:
\begin{itemize}
        \item Abertura da calculadora do Windows (\texttt{calc.exe});
        \item Exposição do computador a ligações remotas sem autenticação;
        \item Execução de binários após a sua transferência via HTTP (sem sucesso).
\end{itemize}

Assim, procedeu-se à examinação do código-fonte, utilizando ferramentas como o \texttt{grep} para procura de expressões regulares no conteúdo, e da secção do programa relativa ao tratamento de agendamentos e datas (onde se dá o \textit{overflow}) -- Schedule.cpp.

Uma vez feita esta busca, os resultados foram inconclusivos uma vez que a localização exacta da porção de código vulnerável foi impossível de determinar.


\pagebreak

\appendix

\pagenumbering{roman}

\section{Análise STRIDE}
\label{sec:stride}

De maneira a melhor explorar as vulnerabilidades presentes, verifica-se que as ameaças mais preponderantes são:

\begin{itemize}
        \item \textit{Spoofing}, pois é possível implantar ficheiros em computadores de outros utilizadores/pessoas;
        \item \textit{Information Disclosure}, dada a possibilidade de nos ser concedido acesso remmoto a dados aos quais não temos direito;
        \item \textit{Tampering with Data}, pois a integridade do programa pode ser comprometida;
        \item \textit{Denial of Service}, pois há situações em que ocorre o crash do programa;
        \item \textit{Elevation of Privilege}, dado que o atacante herda os privilégios do utilizador que activa a \textit{exploit}.
\end{itemize}

\section{SEH Based Overflow -- Passo a Passo}
\label{sec:seh}

Nem sempre é possível fazer uma \textit{stack based exploit}. No entanto, em Windows é possível fazer um ataque com base na SEH (\textit{structured exception handling}). A SEH trata-se de uma lista ligada de registos relativos às excepções de um programa. Mesmo que um programa não tenha \textit{exception handling}, existe um \textit{handler} inerente ao Sistema Operativo Windows que, no caso de não haver outros \textit{handlers}, é actuado.

Cada registo da SEH tem dois ponteiros: um deles aponta para o código a correr no caso da excepção se verificar (SE handler); o outro aponta para a estrutura seguinte da SEH.

A função que é chamada no caso de ocorrer uma excepção tem a particularidade de receber quatro argumentos, sendo que um desses argumentos é a ``\textit{establisher frame}'' cujo valor é o endereço em que se encontra o endereço para a próxima estrutura da SEH. Este endereço é o terceiro elemento no topo da pilha quando a função é chamada. Isto significa que se após a chamada da função for executada uma sequência de instruções POP POP RET, o registo EIP passa a apontar para o endereço imediatamente antes da SE \textit{handler}. Há portanto a possibilidade de se conseguir executar código visto que com o \textit{overflow} é possível controlar tanto o valor da SE \textit{handler} como o valor do ponteiro para a próxima estrutura da SEH.

Os passos necessários para fazer uma SEH \textit{based exploit} estão especificados na Figura~\ref{fig:flowchart1}.

%\begin{enumerate}
%	\item descobrir aproximadamente a partir de que tamanho de input o programa crasha -- exemplo: o programa não apresenta problemas com um input de 1000 caracteres mas sim com um de 2000 
%	\item verificar se o programa é vulnerável a ataques via SEH -- se não for não vale a pena continuar
%	\item determinar o offset da SEH
%	\item encontrar um endereço com uma sequência de instruções POP POP RET
%	\item verificar qual a melhor localização da stack para escrever o shellcode
%	\item escrever as instruções necessária para fazer um ou vários saltos até à localização do shellcode
%	\item escrever o shellcode, sem caracteres terminadores e outros que não seja possível usar
%	\item testar a exploit
%\end{enumerate}

\begin{figure}[h]
        \centering
        \includegraphics[width=0.7\linewidth]{fc1}
        \caption{Fluxograma de uma SEH \textit{exploit}}
        \label{fig:flowchart1}
\end{figure}
De seguida descreve-se brevemente como se pode executar cada passo. O debugger usado é o Immunity Debugger. Outro programa auxiliar é o Metasploit.

\paragraph*{Programa vulnerável a SEH?} Usando o plugin \texttt{mona} é possível determinar facilmente quais os módulos não compilados com SafeSEH. Para tal, após instalação do \texttt{mona} basta executar o seguinte comando:
\begin{lstlisting}
   !mona nosafeseh
\end{lstlisting}
No caso do i.FTP é perceptível que existem dois módulos carregados pelo programa que não têm SafeSEH: \texttt{iftp.exe} e \texttt{Lgi.dll}. Pode-se também ver quais os endereços de memória do seu código.

\paragraph*{Determinar \textit{offset} da SEH} O \textit{offset} da SEH pode ser determinado com o auxílio do comando \texttt{pattern-create} da ferramenta do Metasploit. Esta ferramenta gera uma sequência de bytes sem repetições, em que se torna fácil descobrir o offset de conjuntos de vários bytes.

\begin{lstlisting}
   ruby pattern_create.rb 2000
\end{lstlisting}


2000 deve ser substituído pelo número de bytes que se quiser para a sequência. Esse valor deve ser suficientemente grande para que o programa termine inesperadamente.

Usando o debugger para examinar o \textit{overflow} podemos verificar qual o valor escrito no ``Pointer to the next SEH record'' (nSEH). O \textit{offset} desse valor desde o início do \textit{buffer} pode depois ser obtido com:

\begin{lstlisting}
   ruby pattern_offset.rb Au0A
\end{lstlisting}


em que Au0A é o valor escrito no sítio do ponteiro anteriormente referido para no caso do i.FTP. Ficou-se a saber que o offset do início da estrutura é 600.

\paragraph*{Encontrar POP POP RET} Tem que se encontrar uma sequência POP POP RET num dos módulos que não tenha SafeSEH. Existem várias formas de o fazer. A mais simples é pesquisar uma sequência de comandos no disassembler fazendo \textit{Search for > Sequence of commands (CTRL~+~S)} e escrever nessa janela o que se pode ser na Figura~\ref{find_POPPOPRET}.

\begin{figure}
	\centering
	\includegraphics[scale=1]{find_POPPOPRET}
	\caption{Procurar uma sequência POP POP RET}
	\label{find_POPPOPRET}
\end{figure}

Uma vez encontrada essa sequência há que anotar o seu endereço para o usar mais tarde. Neste caso anotou-se \texttt{0x1001A24A}, pertencente ao \texttt{Lgi.dll}.

\paragraph*{Melhor localização para o \textit{shellcode}} A melhor localização para o código do shellcode varia com a sua dimensão e qual o número de bytes na \textit{stack} que se tem sob controlo. No caso do i.FTP o \textit{offset} do último byte sobre o qual se tem controlo é cerca de 1360 pelo que devemos ter espaço que chegue para o \textit{shellcode} a seguir ao registo da SEH. Deferimos então que o \textit{shellcode} será colocado a seguir ao registo da SEH que está sob nosso controlo.

\paragraph*{Escrever instrução de salto} Para este passo existem várias opções. É possível fazer compilar uma instrução de salto e usá-la. Mas como neste caso se trata apenas de uma instrução optou-se por usar uma referência de x86 para identificar o \textit{opcode} a usar. Só é necessário saltar para o endereço a seguir ao do registo da SEH. Para isso basta um salto de no máximo 8 bytes visto que esse é o tamanho total do registo da SEH. Foram consultados \cite{AMD64vol3_2013} e \cite{refx86asm} e chegou-se à conclusão de que para fazer um near JUMP o \textit{opcode} é \texttt{0xEB} e que o byte seguinte é o valor do salto.

No nSEH escreveu-se por exemplo \texttt{0xEB069090} ou \texttt{0x9090EB04} (\texttt{0x90} são NOPs), ou seja pode-se saltar ou 6 ou 4 bytes dependendo do alinhamento que se der à instrução near JUMP. A seguir à SEH começamos a escrever o \textit{shellcode} e é para lá que se vai saltar.

\paragraph*{Escrever o \textit{shellcode}} Para escrever a \textit{exploit} pode-se usar uma ferramenta do Metasploit, o \texttt{msfpayload}. Este programa gera vários \textit{shellcodes} para diferentes sistemas operativos.

Um exemplo clássico é o de um \textit{shellcode} que abre uma calculadora do Windows:

\begin{lstlisting}
   ruby msfpayload windows/exec CMD=calc.exe R
\end{lstlisting}


No entanto, o \textit{shellcode} gerado pode conter alguns caracteres que não é possível usar (caracteres terminadores de strings) e outros dependendo do caso. Os caracteres terminadores de strings são \texttt{0x00}, \texttt{0x0A} e \texttt{0x0D}. Tem que ser encontrado um \textit{shellcode} alternativo sem esses caracteres, o que pode ser feito com:

\begin{lstlisting}
   ruby msfpayload windows/exec CMD=calc.exe R | ruby msfencode
     -b `\x00\x0A\x0D' -t c
\end{lstlisting}


Verificou-se que no caso do i.FTP outro caracter que parece terminar o overflow é o \texttt{0x26}, pelo que esse caracter também foi excluído.

O output resultante está pronto a ser usado como \textit{shellcode} para o SEH \textit{based overflow}.


\section{Eliminação da Vulnerabilidade}
\label{sec:vulnerability}

Dada a reduzida disponibilidade do grupo e do facto de não lhe ser familiar o código \texttt{C++} com objectos, não foi possível identificar as linhas de código associadas à vulnerabilidade considerada.

No entanto, sabe-se que o \textit{buffer overflow} ocorre devido ao campo \texttt{Time} do ficheiro Schedule.xml. Quando um download é calendarizado no i.FTP, é preenchido um formulário em que campo relativo ao tempo do download é definido através de um menu \textit{dropdown}. O tempo fica assim escrito com a sintaxe correcta. Suspeita-se que a vulnerabilidade é causada em parte ou na totalidade por não ser feita uma verificação do conteúdo do campo \texttt{Time} do ficheiro Schedule.xml que é carregado pelo programa. Se o tamanho e caracteres deste campo fossem verificados, as \textit{exploits} usadas não teriam funcionado porque:

\begin{enumerate}
	\item o conteúdo de \texttt{Time} é claramente demasiado longo para se tratar de uma data válida;
	\item os caracteres injectados neste campo não são representáveis -- por exemplo \texttt{0x10}.
\end{enumerate}


\pagebreak
\bibliographystyle{plain}
\nocite{CorelanTeam, refx86asm, genSEHexploits, AMD64vol3_2013}
\bibliography{CorelanTeam,refx86asm,genSEHexploits,AMD64vol3_2013}	% no spaces between commas!

\end{document}

