\documentclass[a4paper]{article}

\usepackage[portuguese]{babel}
\usepackage{comment}
\usepackage[T1]{fontenc}
\usepackage[utf8]{inputenc}
\usepackage{hyperref}
\usepackage{graphicx}
\usepackage{float}
\usepackage{multirow}
\usepackage[hypcap]{caption} % makes \ref point to top of figures and tables
\usepackage{amsmath}
\usepackage[usenames,dvipsnames,svgnames,table]{xcolor}
\usepackage{rotating}

\begin{document}

\begin{titlepage}

	\begin{center}

		\includegraphics[width=6cm]{./title}\\[3cm]

		\textsc{\LARGE Segurança Informática em Redes e Sistemas}\\[1.5cm]

		\textsc{\Large Projecto}\\[1.5cm]


		{ \huge \bfseries Zero-Day Vulnerability \\[2.5cm] }


		\noindent
		\begin{center} \large
			Gonçalo Ribeiro, 73294\\[5mm]

			António Bacelar de Sousa, 73425\\[5mm]

			Rafael Gonçalves, 73786\\[2.5cm]

		\end{center}

		\begin{minipage}{0.4\textwidth}
			\begin{flushleft} \Large
				Prof. Ricardo Chaves
			\end{flushleft}
		\end{minipage}
		\begin{minipage}{0.4\textwidth}
			\begin{flushright} \Large
				Prof. Miguel Pardal
			\end{flushright}
		\end{minipage}

		\vfill

		{\large \today}

	\end{center}

\end{titlepage}

\tableofcontents
\pagebreak

\section{Motivação}

\pagebreak

\section{Objectivos}
\label{sec:objectivos}

\begin{itemize}
	\item Básico: especificar uma vulnerabilidade no Haihaisoft Universal Player
	\item Intermédio: realização de um ataque ao software por intermédio da vulnerabilidade especificada
	\item Avançado: investigação e aplicação de métodos que permitam resolver total ou parcialmente a vulnerabilidade
\end{itemize}

\subsection{Básico}



\subsection{Intermédio}

\subsection{Avançado}

\pagebreak

\section{Plano de Trabalho}

\subsection{Semana 1}
Prazo: 07/11/14
\subsubsection{Disponibilidade}
Reduzida devido a avaliações e entregas de projectos de outras UCs.

\subsubsection{Tarefas}
\begin{itemize}
\item escolher um software/website com uma vulnerabilidade activa
\item definir os parâmetros iniciais do projecto
\end{itemize}

\subsubsection{Trabalho Realizado}
Nesta semana procurámos listas de software/websites com vulnerabilidades activas e escolhemos o Haihaisoft Universal Player como objecto de estudo deste projecto. Este software é desenvolvido para Windows e é software aberto.

Foram também delineados os objectivos deste projecto (Secção~\ref{sec:objectivos}).

\subsection{Semana 2}
Prazo: 14/11/14
\subsubsection{Disponibilidade}
Reduzida devido a avaliações e entregas de projectos de outras UCs.

\subsubsection{Tarefas}
\begin{itemize}
\item preparar uma máquina virtual que possa ser usada ao longo do projecto
\item testar a \textit{proof of concept} (POC) da vulnerabilidade zero-day
\end{itemize}

\subsubsection{Trabalho Realizado}
Tendo-se verificado que o Haihaisoft Universal Player funciona exclusivamente em Windows foi esse o sistema operativo escolhido para instalar na máquina virtual. Como o Windows é software pago teve-se em conta esse factor de forma a fazer uma instalação que não tivesse problemas com licenças. Neste sentido e visto que o Windows 10 Technical Preview foi lançado recentemente foi esta a versão do Windows que se escolheu instalar na máquina virtual. De seguida instalou-se o Haihaisoft Universal Player e verificou-se o seu correcto funcionamento.

Uma das coisas que nos impressionou foi o facto de que o Haihaisoft Universal Player precisa de privilégios de administrador para ser executado. Como tal, e visto que este software é vulnerável a buffer overflows, o código que conseguirmos injectar vai correr com privilégios de administrador deixando a máquina largamente à nossa mercê.

Por fim testámos a POC encontrada na Exploit Database\footnote{http://www.exploit-db.com/exploits/32514/}. Esta POC consiste num pequeno script escrito em Python. Como tal instalámos Python na máquina virtual. Tentámos correr o script mas o Python revelava alguns erros. Após correcção desses erros conseguimos gerar 3 ficheiros de exploit. Experimentámos então abrir esses ficheiros no Haihaisoft. Os resultados dos 3 ficheiros foram semelhantes. Com um deles o programa fechou-se mal era aberto. Com os outros dois ficheiros o programa deixava de responder (aparecia o menu do Windows a dar essa informação) e passado algum tempo o programa era fechado. O shellcode introduzido é apenas uma instrução, que ainda não tivemos oportunidade de verificar o que faz. No entanto verificámos que claramente algo de errado acontece com o programa ao abrir os ficheiros de exploit.

\subsection{Semana 3}
Prazo: 21/11/14
\subsubsection{Disponibilidade}
Muito reduzida devido ao teste de SIRS.

\subsubsection{Tarefas}
\begin{itemize}
	\item Identificar o objectivo do código Assembly injectado através de buffer overflow
\end{itemize}

\subsubsection{Trabalho Realizado}
%Examinámos o código fonte de maneira a descobrir o endereço de retorno, de maneira a alterarmos o código de exploit para conseguirmos correr código nosso

Nesta semana analisámos a POC. São escritos bytes com 4 valores diferentes para o ficheiro de \textit{exploit}. Chegámos à conclusão de que os valores dos bytes se referem aos caracteres ASCII A, B e C e que o outro valor é usado como \textit{padding}.

\subsection{Semana 4}
Prazo: 28/11/14
\subsubsection{Disponibilidade}
Maior disponibilidade.
\subsubsection{Tarefas}
\begin{itemize}
	\item Desenvolvimento do código que se serve da vulnerabilidade encontrada para conseguir acesso com privilégios elevados.
\end{itemize}

\subsubsection{Trabalho Realizado}
A primeira conclusão a que chegámos foi que não é possível fazer um stack based overflow. O Windows usa SEH (structured exception handlers) o que faz com que stack based overflows não sejam possíveis, ou seja, fazer um overwrite directo do EIP (instruction pointer) e fazer um RET.

Outra conclusão foi que o buffer cuja capacidade é excedida não é directamente o buffer que recebe a URL mas sim um buffer que contém a URL transformada, após transformação/eliminação de caracteres que não são permitidos em URLs. Isto significa que os valores de bytes que nos é possível escrever ficam muito mais limitados. Mais concretamente de 256 valores de bytes passamos a poder usar apenas cerca de 90 valores. Por outro lado não nos é possível usar o valor \texttt{0x00} porque é o terminador de strings e portanto se o usarmos paramos o programa para de copiar bytes assim que encontra um byte com esse valor (parando o nosso overflow antes do que queremos).

O Haihaisoft UMP não foi compilado com protecção da SEH. Ou seja em teoria é possível fazermos overflow das estruturas SEH e enganar o programa de forma a que escreva um novo valor no EIP, valor esse escolhido por nós. Para fazermos isto temos que conseguir que o programa execute uma sequência de instruções POP POP RET que tem que estar em código não compilado com protecção de SEH. Neste caso o próprio executável não tem esta protecção pelo que esta sequência de instruções podia ser uma do próprio programa.

Os programas em Windows têm a memória dividida em 3 componentes. Entre estas, a componente de código é aquela que tem as instruções do programa. Ou seja, é lá que temos que encontrar uma sequência POP POP RET. No entanto, verificámos que os endereços de memória do segmento de código começam todos pelo byte \texttt{0x00}. Isto significa que para enganar o programa a saltar para a dita sequência de código, teríamos que fazer um overwrite da SEH handler para um endereço cujo primeiro byte é \texttt{0x00}. Ora isso não é possível porque não conseguimos escrever esse valor de byte. Portanto não conseguimos que seja executada a sequência de instruções necessária para que conseguíssemos alterar o EIP para uma zona de memória controlada por nós de forma a executar uma exploit.

Decidimos então alterar o alvo de estudo do nosso projecto do Haihaisoft UMP para outro zero-day.


\subsubsection{Novo Paradigma -- i.FTP}
Consultámos o endereço exploit-db.com de forma a encontramos um zero-day novo. O candidato que escolhemos foi o i.FTP. Trata-se de um cliente de FTP que apresenta vulnerabilidades de buffer overflow (decidimos manter o tipo de vulnerabilidades a serem exploradas).

O referido site oferece uma POC para esta vulnerabilidade. Após análise dessa POC concluímos que a vulnerabilidade resulta do carregamento de um ficheiro de configurações do programa -- Schedule.xml. Existe um campo nesse ficheiro que se refere ao tempo para o qual queremos agendar uma transferência. Esse campo é susceptível a overflows.

Fizemos uma análise da vulnerabilidade e conseguimos executar uma exploit (abrir a calculadora do Windows). Esta exploit não é muito interessante e portanto vamos tentar elaborar outra que o seja.

Quando corremos o programa a calculadora é imediatamente aberta e o programa termina com a janela do Windows que diz que o programa não responde. Gostávamos de evitar 	que o programa terminasse de forma a que não seja perceptivel que algo de errado se passou.


\subsection{Semana 5}
Prazo: 05/12/14
\subsubsection{Disponibilidade}
Média.
\subsubsection{Tarefas}
\begin{itemize}
	\item Elaborar uma exploit mais interessante;
	\item Evitar que o programa deixe de responder;
	\item Tentar eliminar a vulnerabilidade;
	\item Conclusão do presente documento.
\end{itemize}


\pagebreak

\section{Resultados}

\subsection{Esperados}
\subsection{Obtidos}

\pagebreak

\section{Referências}

\end{document}